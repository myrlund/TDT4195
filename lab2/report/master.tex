\documentclass[11pt,a4paper]{article}

% For å kunne skrive norske tegn.
\usepackage[utf8]{inputenc}

\usepackage{minted}

% For å inkludere figurer.
\usepackage{graphicx}

% Ekstra matematikkfunksjoner.
\usepackage{amsmath,amssymb}

%\usepackage[section]{placeins}

% \usepackage{hyperref}
% \hypersetup{%
%   colorlinks=true, % hyperlinks will be black
%   urlcolor=red,
%   linkcolor=red
% }

% For å få tilgang til finere linjer (til bruk i tabeller og slikt).
%\usepackage{booktabs}

% For justering av figurtekst og tabelltekst.
%\usepackage[font=small,labelfont=bf]{caption}

% Subsections A, B,
\renewcommand{\thesection}{\Roman{section}}
\renewcommand{\thesubsection}{\arabic{subsection}}

% Disse kommandoene kan gjøre det enklere for LaTeX å plassere figurer og tabeller der du ønsker.
\setcounter{totalnumber}{5}
\renewcommand{\bottomfraction}{0.95}
\renewcommand{\floatpagefraction}{0.35}

\begin{document}

  % Rapportens tittel:
  \title{Lab 2 \\ \large{TDT4195: Grunnleggende Visuell Databehandling}}
  \author{Jonas Myrlund}

  % Her ber vi LaTeX om å lage tittelen (til nå har vi bare sagt hva den skal inneholde):
  \maketitle
  
  \section{Change RenderScreen5} % (fold)
  \label{sec2}
  
    To get the top left corner involved, I changed the vertex matrix to the following (note the awesome new general shape):

    \begin{minted}[gobble=6]{c++}
      static const GLfloat g_vertex_buffer_data[] = {
          -1.0f,   1.0f, 0.0f,
          1.0f,   -1.0f, 0.0f,
          -0.75f,  1.0f, 0.0f,
      };
    \end{minted}

    In order for it to turn blue, I changed the SimpleFragmentShader to the following:

    \begin{minted}[gobble=6]{glsl}
      #version 330 core

      // Ouput data
      out vec3 color;

      void main()
      {

        // Output color = blue 
        color = vec3(0,0,1);

      }
    \end{minted}
  
  % section sec2 (end)

  \section{Create RenderScreen6} % (fold)
  \label{sec:create_renderscreen6}
  
    We'll reuse the shader from RenderScreen5, and simply extend the \texttt{g\_vertex\_buffer\_data} array with the required vertices.

    To make the scene load, I also had to perform some straight-forward changes around the project, but I won't detail them here.

    The matrix now looks like this:

    \begin{minted}[gobble=6]{c++}
      static const GLfloat g_vertex_buffer_data[] = {

          // Scene 5
          -1.0f, 1.0f, 0.0f,
          1.0f, -1.0f, 0.0f,
          -0.75f,  1.0f, 0.0f,

          // Scene 6
          -1.0f, -1.0f, 0.0f,
          0.0f,   0.0f, 0.0f,
          0.0f,  -1.0f, 0.0f,

          -1.0f, 1.0f, 0.0f,
          0.0f,  0.0f, 0.0f,
          0.0f,  1.0f, 0.0f,

          1.0f, 1.0f, 0.0f,
          1.0f, -1.0f, 0.0f,
          0.0f, 0.0f, 0.0f,

      };
    \end{minted}

    And the relevant code in RenderScreen6 looks like this, working from an offset of 3 vertices:

    \begin{minted}[gobble=6]{c++}
      glDrawArrays(GL_TRIANGLES, 3, 9);
    \end{minted}

    Otherwise, RenderScreen6 looks just like RenderScreen5.

  % section create_renderscreen6 (end)

  \section{Minor code tasks} % (fold)
  \label{sec1}

    \subsection{Center window} % (fold)
    \label{sub:center_window}
      
      The $x$ position is half the window's width to the left of the middle of the screen.
      Likewise for $y$.

      \begin{minted}[gobble=8]{c++}
        int iWindowWidth = 512;
        int iWindowHeight = 512;
        
        glutInitWindowPosition( iScreenWidth / 2 - iWindowWidth / 2,
                                iScreenHeight / 2 - iWindowHeight / 2 );
        glutInitWindowSize( iWindowWidth, iWindowHeight );
      \end{minted}

    % subsection center_window (end)
  
    \subsection{Render a torus} % (fold)
      \label{sub:render_a_torus}
      
        Replacing the rendering call in RenderScreen1 with:

        \begin{minted}[gobble=8]{c++}
          glutSolidTorus(10.0, 20.0, 100, 200);
        \end{minted}

      % subsection render_a_torus (end)  

    \subsection{Explain why the Scene 1,2,3 and 4 are displayed wrong after the user enable scene 5} % (fold)
    \label{sub:explain_why_the_scene_1_2_3_and_4_are_displayed_wrong_after_the_user_enable_scene_5}
    
      The \texttt{glUseProgram(shaderProgramId)} call says that we want to use the given shader, as we do in scene 5.
      However, the reset call, \texttt{glUseProgram(0)} is never called, so the shader from 5 still applies.

      Calling \texttt{glUseProgram(0)} whenever switching between scenes solves this problem.

    % subsection explain_why_the_scene_1_2_3_and_4_are_displayed_wrong_after_the_user_enable_scene_5 (end)
    
  % section sec1 (end)

\end{document}















