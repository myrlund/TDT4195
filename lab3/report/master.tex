\documentclass[11pt,a4paper]{article}

% For å kunne skrive norske tegn.
\usepackage[utf8]{inputenc}

\usepackage{minted}

% For å inkludere figurer.
\usepackage{graphicx}

% Ekstra matematikkfunksjoner.
\usepackage{amsmath,amssymb}

%\usepackage[section]{placeins}

% \usepackage{hyperref}
% \hypersetup{%
%   colorlinks=true, % hyperlinks will be black
%   urlcolor=red,
%   linkcolor=red
% }

% For å få tilgang til finere linjer (til bruk i tabeller og slikt).
%\usepackage{booktabs}

% For justering av figurtekst og tabelltekst.
%\usepackage[font=small,labelfont=bf]{caption}

% Subsections A, B,
\renewcommand{\thesection}{\Roman{section}}
\renewcommand{\thesubsection}{\arabic{subsection}}

% Disse kommandoene kan gjøre det enklere for LaTeX å plassere figurer og tabeller der du ønsker.
\setcounter{totalnumber}{5}
\renewcommand{\bottomfraction}{0.95}
\renewcommand{\floatpagefraction}{0.35}

\begin{document}

  % Rapportens tittel:
  \title{Lab 3 \\ \large{TDT4195: Grunnleggende Visuell Databehandling}}
  \author{Jonas Myrlund}

  % Her ber vi LaTeX om å lage tittelen (til nå har vi bare sagt hva den skal inneholde):
  \maketitle
  
  \section{Rotate the cube in X} % (fold)
  \label{sec1}
  
    To get the top left corner involved, I changed the vertex matrix to the following (note the awesome new general shape):

    \begin{minted}[gobble=6]{c++}
      static const GLfloat g_vertex_buffer_data[] = {
          -1.0f,   1.0f, 0.0f,
          1.0f,   -1.0f, 0.0f,
          -0.75f,  1.0f, 0.0f,
      };
    \end{minted}

  
  % section sec1 (end)

\end{document}















